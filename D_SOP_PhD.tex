\documentclass[12pt]{article}
\setlength{\oddsidemargin}{0in}
\setlength{\evensidemargin}{0in}
\setlength{\textwidth}{6.5in}
\setlength{\topmargin}{-.3in}
\setlength{\textheight}{9in}
\pagestyle{empty}
\usepackage{hyperref}
\usepackage{graphicx}
\usepackage{amsfonts}
\usepackage{amsmath}
\usepackage{url}
\usepackage[toc,page]{appendix}
\usepackage{listings}
\usepackage{booktabs}

\begin{document}

\begin{center}
{\Large Statement of Purpose} \\[.1in]
{\large Surya Teja Eada} 
\end{center}

\vspace*{.5in}

\text It is the pursuit of working in interdisciplinary fields and applying my knowledge in statistics to real world problems that led me to consider the choice of applying to the prestigious Doctoral Program in Statistics at the University of Florida. I wish to pursue my doctorate study and establish myself in the career of research and teaching. Therefore, I submit my Statement of Purpose where I have described how my academic background, professional experience and my successes and failures have shaped my interest towards research in Statistics.

Sharing knowledge gives me immense satisfaction and there have been many instances in my life that have proven this point to me. While I was in school, I loved Mathematics and enjoyed solving challenging problems put forth by my friends. Similarly while pursuing undergraduate studies in Statistics, I inculcated the habit of discussing complex problems in the field of Statistics. Currently I work in the field of Risk Management, where I validate the appropriate usage of Statistics in credit risk models and I love the way my knowledge in the field of Statistics is being utilized in the real world. This has spiked my interest to pursue the field of Statistics further and to know and utilize various techniques for application of Statistics. This brings me back to the pursuit of learning and doing research in statistics, the subject I adored all through my academic career at the Indian Statistical Institute.

My never-ending thirst and interest in mathematics was majorly due to my parents. My parents were both teachers of Mathematics before they became administrators of a school that they had established. This gave me an early head start in the field of mathematics. For example, my parents taught me multiplication before my classmates ever knew of multiplication. Instances like this boosted my interest in mathematics and kept me ahead of my classmates in solving problems. Over years I have ensured that this reputation never dies down by making sure that I am always ahead in solving difficult mathematical problems. 

Interest in mathematics has also been a guideline for some of the most important decisions in my life. The choice of Statistics as the field of undergraduate study is one such example. While I had been selected in two premier institutes in India, Indian Institute of Technology and Indian Statistical Institute, my interest in mathematics justified my choice of selecting Indian Statistical Institute (ISI). This paved the way for me to be one of the 30 candidates chosen ahead of thousands of aspirants at the most prestigious research institute of India in the field of Mathematics and Statistics. To top it off, I also received the national level scholarship, INSPIRE funded by the Department of Science and Technology, India.

Soon after joining ISI, I was overwhelmed by the depth at which Statistics was taught at ISI. There was a vast difference between Probability theory, matrix algebra, analysis taught at school and the same subjects taught at ISI. My knowledge of mathematics prior to ISI seemed so minuscule. I could understand and solve problems like I did earlier but could not excel in scoring high marks during the exams. I was confused and did not realize the reason behind my failure early enough for me to make amends. It worsened in the next semester when I did not understand how probability was defined in a continuous space. As a result, I scored less than required marks (35) in the subject. This was an eye opener for me to change my methods of learning subjects at ISI. I started practicing more problems, visualizing how theorems are applied and have started working diligently in order to excel in my field. I scored 99 out of 100 when I retook the exam. Though I realized that the maximum obtainable marks when you retake an exam are 45, I was happy for I knew that I excelled in the subject of probability. My academic career's graph took a turn from that day and I have improved my percentage with each passing semester when I also kept constantly bridging the knowledge gap that was created in the first few semesters. I was introduced to Biology and applications of statistics in biology when I chose Biology as one of my elective subjects between semesters 3 to 5 of my B.Stat program. The Biology curriculum for these semesters contained topics of molecular biology, agricultural science and anthropology.

From the second year up until the last year, I opted to use my summer vacations for various projects that gave me scope to learn and contemplate how mathematics and statistics are applied in the real world. I spent my second summer in the C.R.Rao Advanced Institute of Mathematics, Statistics and Computer Science to hear lectures and gain knowledge in a new field, Cryptography. Here, I was exposed to the utility of number theory, algebraic fields in order to encrypt and decrypt messages. I was thrilled when I knew the utility of this subject dated back to World War 2 where enigma machines were used. My first stint at applying statistics on a problem was during the course of a project undertaken in the final year of my B.Stat program. The objective of the project was to evaluate if pollution has significant effect on length of fish. We used the growth curve modeling of fish and the ARMAX modeling as the two different approaches in order to evaluate and test for the significance of pollution on growth of fishes. 

I also opted to use my next couple of summer vacations after the final year of Stat and 1st year of M.Stat to do internships at eClerx and Deloitte U.S. India Ltd respectively. As an intern at eClerx, I was given the task to create training decks in regression methodology and cluster analysis to be used for the employees who would join the firm. This gave me an opportunity to re-visit all my prior courses in statistics and also introduced me to few new concepts such as Cluster Analysis. It was during this stretch that I took on the role of a trainer and realized my interest to share knowledge. In Deloitte, I was introduced to supervised learning techniques such as CHAID decision tree algorithm, which I used to perform market segmentation to target customers for purchasing particular electronic devices. 

I chose Actuarial Statistics as my specialization in the Final year of M.Stat. This gave scope for me to learn survival models, actuarial models and life contingency models which all exposed me to the applications of Statistics in the actuarial fields. After finishing my graduation course, I chose to take up a job in the financial risk management industry.  The knowledge that I gained in actuaries regarding loss distributions, Probability of Default, Loss given default have all come handy and helped me during my tenure at CRISIL Global Research & Analytics. 

My job responsibility at CRISIL includes evaluating the right usage of statistics and validating a particular U.S bank's credit risk models. During this tenure, I validated different types of statistical models used for the bank that include Logistic modeling of Probability of default, Generalized linear regression for Loss given default, Monte Carlo simulation for Operational Risk, and Monte Carlo simulation for Credit value adjustment model. I worked diligently going through research papers in Statistics and Finance, and also banking regulations. The research papers that I went through induced in me the interest to do research and contribute to interdisciplinary research fields involving statistics. The company recognized my work at job and I was awarded 'Bright Spark' and 'Execution Excellence' awards in two consecutive quarters in 2015. I also obtained a 'highly proficient' rating during my performance appraisal.

The rigorous curriculum at ISI, professional experience and various projects have built my interest in the following fields; Linear Algebra, Stochastic Models, Statistical Inference, Decision theory, Survival Modeling, Generalized linear modeling, Regression techniques, Categorical data analysis. After going through the department website of North Carolina State University, Statistics, I have found many current research topics very appealing. This spiked my desire to associate myself with your university in doing research. I will be grateful if I am provided an opportunity to pursue research. I am willing to make a whole-hearted effort in contributing to the field of Statistics. 

I am thankful to you for considering my application.

\end{document}

