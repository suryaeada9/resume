\documentclass[a4paper,10pt]{article}

%A Few Useful Packages
\usepackage{marvosym}
\usepackage{fontspec} 					%for loading fonts
\usepackage{xunicode,xltxtra,url,parskip} 	%other packages for formatting
\RequirePackage{color,graphicx}
\usepackage[usenames,dvipsnames]{xcolor}
\usepackage[big]{layaureo} 				%better formatting of the A4 page
% an alternative to Layaureo can be ** \usepackage{fullpage} **
\usepackage{supertabular} 				%for Grades
\usepackage{titlesec}					%custom \section

%Setup hyperref package, and colours for links
\usepackage{hyperref}
\definecolor{linkcolour}{rgb}{0,0.2,0.6}
\hypersetup{colorlinks,breaklinks,urlcolor=linkcolour, linkcolor=linkcolour}

%FONTS
\defaultfontfeatures{Mapping=tex-text}
%\setmainfont[SmallCapsFont = Fontin SmallCaps]{Fontin}
%%% modified for Karol Kozioł for ShareLaTeX use
\setmainfont[
SmallCapsFont = Fontin-SmallCaps.otf,
BoldFont = Fontin-Bold.otf,
ItalicFont = Fontin-Italic.otf
]
{Fontin.otf}
%%%

%CV Sections inspired by: 
%http://stefano.italians.nl/archives/26
\titleformat{\section}{\Large\scshape\raggedright}{}{0em}{}[\titlerule]
\titlespacing{\section}{0pt}{3pt}{3pt}
%Tweak a bit the top margin
%\addtolength{\voffset}{-1.3cm}

%Italian hyphenation for the word: ''corporations''
\hyphenation{im-pre-se}

%-------------WATERMARK TEST [**not part of a CV**]---------------
\usepackage[absolute]{textpos}

\setlength{\TPHorizModule}{30mm}
\setlength{\TPVertModule}{\TPHorizModule}
\textblockorigin{2mm}{0.65\paperheight}
\setlength{\parindent}{0pt}

%--------------------BEGIN DOCUMENT----------------------
\begin{document}

%WATERMARK TEST [**not part of a CV**]---------------
%\font\wm=''Baskerville:color=787878'' at 8pt
%\font\wmweb=''Baskerville:color=FF1493'' at 8pt
%{\wm 
%	\begin{textblock}{1}(0,0)
%		\rotatebox{-90}{\parbox{500mm}{
%			Typeset by Alessandro Plasmati with \XeTeX\  \today\ for 
%			{\wmweb \href{http://www.aleplasmati.comuv.com}{aleplasmati.comuv.com}}
%		}
%	}
%	\end{textblock}
%}

\pagestyle{empty} % non-numbered pages

\font\fb=''[cmr10]'' %for use with \LaTeX command

%--------------------TITLE-------------
\par{\centering
		{\Huge Surya Teja Eada 
	}\bigskip\par}

%--------------------SECTIONS-----------------------------------
%Section: Personal Data
\section{Personal Data}

\begin{tabular}{rl}
    \textsc{Address:}   & 108, Foster Drive, Willimantic, Connecticut - 06226 \\
    \textsc{Phone:}     & +1 860-208-7249\\
    \textsc{email:}     & \href{mailto:teja.suryateja.surya@gmail.com}{teja.suryateja.surya@gmail.com}
\end{tabular}

%Section: Work Experience at the top
\section{Work Experience}
\begin{tabular}{r|p{11cm}}
 \emph{May 2017} & Senior Research Analyst at \textsc{CRISIL Global Research and Analytics}, \\\textsc{July 2014}&Pune \\&\emph{Consultant,  Model Validation for Risk related Banking models}\\&\footnotesize{As a credit risk model validator, I have used statistics in the field of Financial Risk Management. My work predominantly involved validation of credit risk models. I have extensive experience, having worked with Probability of Default (Logistic Regression), Loss Given Default (Fractional Logistic), Operational Loss Forecasting (Distribution Fitting and Monte Carlo Simulations), Automated Valuation Models' Performance Assessment Model (Wilcoxon Sign Rank and MAE based ranking model), Economic Capital Models (Simulation Based Vendor Model). I worked as consultant for one of the leading retail banks in the United States of America.}\\\multicolumn{2}{c}{} \\
 \textsc{Summer 2013} & Summer Internship at \textsc{Deloitte Touche Tohmatsu India LLP.}, \\ & Hyderabad \\&\emph{Advanced Analytics and Modeling}\\&\footnotesize{I have developed a supervised learning decision tree (CHAID) model to deduce significant categorical factors that impacts purchase of a particular electronic good. The model was built for the purpose of obtaining prospective target population based on the significant characteristics such as Age, purchase behavior in different locations.}\\\multicolumn{2}{c}{} \\
\textsc{Summer 2012} & Summer Intern at \textsc{eClerx}, Mumbai \emph{}\\&\footnotesize{I created training presentations on statistical techniques such as linear regression and clustering.}
\end{tabular}

%Section: Education
\section{Education}
\begin{tabular}{rl}	
 \textsc{Current}  & Master's in \textsc{Applied Financial Mathematics}, (Graduation: May 2019) \\
 & \textsc{and, TEACHING ASSISTANT for Calculus 2 Undergraduate Course} \\
 & \textsc{Department of Mathematics}, \textbf{University of Connecticut} \\
&\normalsize \textsc{GPA}: A \\&\\
 \textsc{May} 2014 & Master of \textsc{Statistics}, \textbf{Indian Statistical Institute}, Kolkata\\
 & (Actuarial Specialization) \\
&\normalsize \textsc{Percentage}: 62.7 \\&\\
\textsc{May} 2012 & Bachelor of \textsc{Statistics} (\textbf{Hons.}), \textbf{Indian Statistical Institute}, Kolkata\\ 
&\normalsize \textsc{Percentage}: 62.3 \\&\\

\end{tabular}

\section{Computer Skills}
\begin{tabular}{rl}
 Good Knowledge:& \textsc{SAS}, \textsc{R} , \textsc{C},  \textsc{MATLAB}, my\textsc{sql}, and \textsc{MS Office} \setmainfont[SmallCapsFont=Fontin-SmallCaps.otf]{Fontin.otf}\\

Basic Knowledge:& \textsc{Python}, and \textsc{Latex}\\
\end{tabular}



%\newpage
%\hypertarget{gmat}{\textsc{Gmat}\setmainfont{LMRoman10 Regular}\textregistered\setmainfont[SmallCapsFont=Fontin-SmallCaps]{Fontin-Regular}}

%\XeTeXpdffile ''GMAT.pdf'' page 1 scaled 800

\end{document}
