\documentclass[12pt]{article}
\usepackage[left=1in,top = 1in,right=1in,bottom = 1in]{geometry}
%\usepackage[letterpaper, hmargin=1in,vmargin=0.75in]{geometry}
%\usepackage[left=2.5cm,top=3cm,right=2.5cm,bottom=3cm,bindingoffset=0.5cm]{geometry}
\usepackage{amsmath}
\usepackage{amssymb}
\usepackage{amsfonts,latexsym}
\usepackage{graphicx}
\usepackage{multirow}
\usepackage{layout}
%\usepackage{fancyhdr}
%\setlength{\headheight}{15.2pt}
%\pagestyle{fancy}
\def \mbZ {\mathbb{Z}}



\begin{document}

\title{Personal Statement}
\author{{\bf Surya Teja Eada} \\  {\small Applying to the PhD Program } \\ {\small Department of Mathematics and Statistics, University of Massachusetts, Amherst } }
\maketitle

\begin{flushcenter}

{\it Education is the most powerful weapon which you can use to change the world. $-$ Nelson Mandela}
\\
\end{flushcenter}

Today's world is constantly changing as a result of decisions that are mostly driven by data. Hence, Statistics education is becoming more important than ever and Statisticians are having better opportunities in every field. However, to ensure that best Statistical techniques are applied in all fields, Statistics departments need to treat this surge in demand as a responsibility, more than just an opportunity. \\

Steadily, Applied Statistics is becoming a necessary utility, but rigorous Statisticians are few in number. As a consequence, few researchers from other fields and industry employees are sometimes applying quick-learned threadbare statistical techniques in their quantitative analysis. For example, when I was working in the Finance industry, I observed a practice of fitting Poisson distribution for ``frequency of losses in a year" and stating the reason that it is a standard industry practice. Being the model validator, when I inspected the variable, it had a variance quite larger than the mean. This implied that the model was underestimating its losses and thus the bank was underestimating its capital requirement. I realized how this could jeopardize a whole industry or in this case, even the economy. This motivated me for having a long term career goal to be a researcher who could collaborate with other researchers and map the best statistical practices with various applications in each field. \\

On the other hand, I also intend to be able to teach Statistics and foster the next generation Statisticians. I realized that I learn a lot and also get new ideas when I am teaching to friends, colleagues, or students. I think that the task of teaching not only gives me immense satisfaction but also has the ability to boost the chances of mapping the best statistical practices with various applications as we reiterate various topics of Statistics. \\ 

To accomplish my goals, I wish to learn and apply various Statistical techniques in different fields such as Biology, Finance, Economics, Insurance, Pharmacy, Chemistry etc. While I was at the University of Connecticut, I talked to researchers from fields of Statistics, Chemistry, Pharmacy, Computational Biology, Plant Science, Communications, Actuaries and more and found that Statistics is widely used in all these fields. Further, I realized that universities in the USA such as the University of Massachusetts have multiple departments and great minds working on research in every field applying Statistics. The work being done by faculty in your department such as Patrick Flaherty in Bio-Statistics, Computational Statistics and Krista J Gile in behavioral science research, social network structures attracts me a lot to your department. Further Statistical Consulting Services in the department also allows for great collaborations with different fields which will enhance chances of me meeting my long term goal. Thus, I strongly believe that pursuing Ph.D. from the University of Massachusetts, Amherst will allow great collaborations and enhance my chances to fulfill my agendas strategically to meet my goal. \\

Having detailed why pursuing a Ph.D. will benefit me, I also want to let you know what makes me feel confident of having the skills to be an active researcher who can contribute to your department and also the field of Statistics. I've been one of the top 30 candidates chosen ahead of thousands of aspirants at the most prestigious research institute of India in the field of Mathematics and Statistics, Indian Statistical Institute (ISI). \\

Indian Statistical Institute has a rigorous curriculum dedicating five years to teaching many Statistical and Applied Statistics courses.  The institute gives great respect to Statistics by laying a very strong foundation with courses in Linear Algebra, Analysis and Probability theory. Advanced topics such as Statistical Inferences, Regression Techniques, Time Series, and Stochastic Processes are discussed in such detail that students are meticulous while applying Statistical techniques in any models. I have personally realized how meticulous I was compared to my peers at applying Statistics during my internships and professional work that spanned for more than three years. \\

During my profession as a Model risk validator, I got to work on projects utilizing parametric and non-parametric statistical techniques such as parameter estimation, hypothesis testing, clustering analysis, logistic regression, Wilcoxon rank sum tests, Markov Models, distribution fitting, and Simulations. Working on these areas in Statistics have kept me constantly engrossed and curious and thus I find applications of these as my areas of interest. Recently, I also started working with Professors Jun Yan and Vladimir Pozdnyakov from Department of Statistics, University of Connecticut on parameter estimation in a Stochastic Process that resembles a Hidden Markov Chain. \\

The passion shown on working in the recent Statistics project while taking courses for my Masters and being a teaching assistant made me feel confident that I can pursue research as a Ph.D. candidate. At the University of Connecticut, I held different roles being a teaching assistant as a discussion leader, online help, grader, and main instructor for the Department of Mathematics. One thing I have learned from exposure to the industry is to be able to self assess reasonably. When I assess myself, I find that I have the necessary statistical knowledge and the ability and dedication to take more advanced courses when required to improve my repertoire. I hereby submit my application with the sincerest hope and belief that it will be considered favorably by your department. \\



%\fancyfoot[C] {me}

%\begin{flushleft}

%Surya Teja Eada,  \\
%University of Connecticut. \\

%\end{flushleft}


\end{document}