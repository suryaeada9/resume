\documentclass[a4paper,10pt]{letter}

\usepackage[utf8]{inputenc}
\usepackage[margin=1in]{geometry}

\begin{document}

\begin{letter}{%
Department of Statistics \\
University of Connecticut \\
Storrs, CT - 06268
}

\opening{Dear Professor Chen,}

I'm eager to apply for the Pre-doctoral Fellowship at the Department
of Statistics, University of Connecticut. With four years as a Ph.D.
student in Statistics and six years of teaching experience at the
university, I have refined my skills, particularly in teaching and
instruction. This fellowship presents the ideal opportunity for me to
intensively pursue my research during my final semester and engage
even more in scholarly activities.

During my time at the University of Connecticut's Department of
Statistics, I've consistently shined in my academic pursuits. I
cleared my qualifiers on the first attempt in 2020 and passed my
generals in January 2023. My dedication has been recognized through
awards such as the H. Fairfield Smith and Dolores S. Smith Award for
Excellence in Applied Statistics, and the Certificate of Excellence
for the Best Performance in Probability. After my second year, I even
secured an internship at IMSI to work in leaf modeling, which further
enhanced my practical knowledge in Statistics. These achievements
guided me to my current research area, focused on continuous-time
stochastic processes. I'm particularly intrigued by the processes
driven by the telegrapher's method, inspired by similar research in
animal movement by my advisors, Vladimir Pozdnyakov and Jun Yan.

The cornerstone of my thesis concentrates on the estimation of
parameters of continuous stochastic processes that are driven by the
telegrapher's process. Our initial work utilizes two Brownian
processes with distinct drifts, which elucidates specific stock price
patterns by transitioning between two hidden states, representing
economic fluctuations. Within these states, increments follow distinct
Brownian processes with their own drifts and volatility.  We've
developed a maximum likelihood estimation (MLE) method tailored for
this process utilizing dynamic programming approaches. The overall
process is observed to be effective for data points with irregular
intervals—something that sets it apart from conventional discrete
models like ARMA-GARCH, which requires evenly spaced observations. 

The insights from this research were presented at the New England
Statistics Symposium (NESS).  Currently, our work is also under peer
review for publication in ``Statistical Inference for Stochastic
Process".  We also explored an approximation to the numerical
integrals for high-frequency data to enhance and speed up our
analysis, potentially forming a second chapter. Our current focus lies
in the Meixner process, a type of L\'evy process. This exploration
offers insights into the presence of high kurtosis over shorter time
spans, a feature beneficial for numerous applications. We've also
conducted simulations to examine the Meixner Brownian process
influenced by telegrapher's state process, particularly in relation to
animal movement. My current research is crucial for advancement of
continuous regime switching state-space processes.

With the additional time the fellowship provides, I plan to deepen my
research and present my findings at various conferences and seminars.
I remain committed to active peer engagement and fostering an
atmosphere of academic distinction through increased scholarly
pursuits. Included with this letter are my curriculum vitae (CV), an
unofficial transcript, the first paper currently under review. I would
welcome the opportunity to discuss my candidacy further and elaborate
on how my research is progressing.  Thank you for your consideration.

Sincerely, \\
Surya Teja Eada. \\ 
\end{letter}

\end{document}
