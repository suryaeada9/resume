\documentclass[10pt]{article}
\usepackage[left=1in,top=0.75in,right=1in,bottom=0.5in]{geometry}
%\usepackage[letterpaper, hmargin=1in,vmargin=0.75in]{geometry}
%\usepackage[left=2.5cm,top=3cm,right=2.5cm,bottom=3cm,bindingoffset=0.5cm]{geometry}
\usepackage{amsmath}
\usepackage{amssymb}
\usepackage{amsfonts,latexsym}
\usepackage{graphicx}
\usepackage{multirow}
\usepackage{layout}
%\usepackage{fancyhdr}
%\setlength{\headheight}{15.2pt}
%\pagestyle{fancy}
\def \mbZ {\mathbb{Z}}



\begin{document}

\title{Statement of Academic Purpose}
\date{}
\author{{\bf Surya Teja Eada} \\  \ \ \\ {\small surya.eada@outlook.com }\\ {\small Applying to the PhD Program } \\ {\small Department of Statistics, ##University ##Name## } }

\maketitle

\begin{flushright}

{\it “Knowledge increases by sharing, but not by saving.}

\end{flushright}


I earnestly wish to apply for admission to the Ph.D. program in Statistics at University of Connecticut, as I’d like to establish myself in career of research and academics in Statistics. I firmly believe that when you teach or discuss, you also learn, because different people can bring different perspectives to the table even in a simple topic. I also believe that this is a major reason why exposure to diverse population such as in the universities in United States have so much success. Also, many instances from my childhood made me realize, the inherent passion I have for teaching, and learning from it.  \\

I constantly pursue doing projects where I could see application of Statistics in the real world. Even in my current occupation as a credit risk model validator, I validate the models that predict financial credit losses where Statistics is fascinatingly used to determine loss forecasts based on the intrinsic volatility in the financial markets and the possibility of a customer’s default. Research for many of these projects led me to understand the various statistical techniques that are applicable along with their strengths and limitations. During my model validations, I have improved many models ensuring that there was no usage of inappropriate statistical techniques for the sake of lucidity. I have always dreamt of building models that are simple to understand and also very accurate with very few limitations. This is when I realized my deep desire to pursue higher levels in Statistical education with which I can contribute to the field of Applied Statistics.\\

I have been extremely interested in mathematics since childhood probably that I can attribute to my parents being teachers in mathematics. Interest in mathematics paved the way for me to be one of the 30 candidates chosen ahead of thousands of aspirants at the most prestigious research institute of India in the field of Mathematics and Statistics, Indian Statistical Institute (ISI). To top it off, I also received the national level scholarship, INSPIRE funded by the Department of Science and Technology, India. \\

I believe I am lucky to be a part of ISI fraternity, as subjects such as Linear Algebra, Analysis, and Probability Theory on which the very foundation of Statistics is built were taught to us rigorously over the span of first two years. Further Statistics, Statistical Inference, and advanced courses such as Stochastic Processes, Complex Analysis, Measure Theory, Multivariate Analysis and Survival Models were taught in the later years in an exemplary manner by the excellent faculty of Indian Statistical Institute. \\

While the first years at ISI laid a strong foundation in Mathematics and Statistics, the last years of ISI curriculum was of great use through-out my academic career and employment career when I was applying Statistics in various projects. It is surprising that I even find my arguments at most times constructed using data in a similar manner to Statistical Inferences. Until now, I have worked through various diverse statistical projects through my academic and professional career. \\

I strongly believed in learning through experience. In my first project at ISI, I had performed hypothesis testing under modeling scenarios such as ARMAX and growth curve modeling, to explain effect of pollution on physical growth of fishes. I also opted to use my next couple of summer vacations to do internships at eClerx and Deloitte U.S. India Ltd. It was these internships that made me understand limitations of my know-how in terms of applications of Statistics in the real world. I have also been introduced to complex techniques such as clustering analysis and CHAID decision trees in its application in cross-sale market segmentation. These projects and internships gave me scope to learn and contemplate how statistics is utilized for various applications in different fields in practice. \\

I chose Actuarial Science Specialization in the last year of Master of Statistics at ISI as it was based on application of Statistics in the actuarial field. Rightly, the subjects such as Survival models, Actuarial models have helped me understand many applications of Statistics in the actuarial fields as well as finance industry. When I joined CRISIL after my masters, I had the opportunity of observing the similarities between calculating premiums based on survival rate during actuarial studies and pricing derivative instruments based on default rate of counter-party.  \\

At CRISIL, I have worked on my domain knowledge in Finance and Risk Management along with technical skills. But my love towards Statistics meant that I could majorly observe Statistical information from the research papers that I read for various projects. During this, I could only think of many other alternative approaches or statistical techniques with fewer limitations that can be applied. I constantly share my knowledge with my colleagues and discuss alternative statistical techniques more apt for the kind of data observed. I do mock projects owing to my interest. Few examples were modeling of a fractional outcome using inflated beta distribution, simulations from a correlated bivariate normal distributions, and also evaluating price of an option based on Geometric Brownian motion simulations of an underlying stock price. \\

I strongly believe that I have the necessary skills and the zeal to be successful in the doctorate studies and contribute to the field of Statistics. Five years of rigorous curriculum at ISI and my professional background ensured that I develop analytic thinking and a strong foundation in Mathematics, Statistics and Programming. It also led to interest in the following fields; Stochastic Modeling, Time Series, Tree algorithms, Nonparametric Statistical Inference, Decision theory, Survival Modeling, Regression techniques, Large sample data analysis, Network Models and Machine Learning.  I found the graduate course curriculum from the department’s website very close to my research interests. I believe that the curriculum along with the faculty from your department will boost my pursuit of research enabling me to excel in the field of Statistics. I hereby submit my application with the sincerest hope and belief that it will be considered favorably by your department. \\

%\fancyfoot[C] {me}

\begin{flushleft}

Surya Teja Eada.\\
\ \ \\
CRISIL Global Research and Analytics, Pune.\\
teja.suryateja.surya@gmail.com

\end{flushleft}


\end{document}